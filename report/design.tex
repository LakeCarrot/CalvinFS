\section{System Design}

Our adaptive remastering mechanism is based on the CalvinFS~\cite{thomson2015calvinfs} system. CalvinFS leverages the features of Calvin, which avoids ordering operations when holding the locks to manage metadata storage for file system. However, low throughput stemmed from the long RTT between geo-distributed replicas requires us to adjust the system architecture to trim the cross-replica communication. In general, the key insight is that we assign each file with a local master replica to manage the read/write operations of this file, and transactions involved with only locally-mastered metadata will be serialized locally, without a cross-replica Paxos application to achieve global ordering. Hereby, we name our modified CalvinFS system as \name.

\subsection{System Components}
\name{} consists of following key components, \textsf{Scheduler}, \textsf{BlockLog}, \textsf{MetaStore}, and \textsf{PaxosApp}. As shown in Figure~\ref{fig:arch}, clients submit requests to \textsf{BlockLog}, which parses the transactions and distribute them on corresponding machines. Then it submits a batch of transactions to \textsf{PaxosApp} for ordering. \textsf{Scheduler} on each machine will fetch the sequence of all transactions from the PaxosApp in local replica, and then call the \textsf{MetaStore} to execute the the transaction. In the following, we will delve into details about \textsf{Scheduler}, \textsf{BlockLog}, and \textsf{PaxosApp}, which carries the essential logic of our geo-distributed optimized \name{} system.

\heading{BlockLog} This part, we explain the function of it. 

\heading{Scheduler} This part, we explain the locking $\&$ fetch $\&$ and reroute/detect.

\heading{PaxosApp} This part we explain how does it serialize and coordinate with each other

\subsection{Remaster Protocol}
\note{blue}{Draw a figure to show the sequence of event during a Remaster operation. And in this subsection, we use a simple example to show how we deal with the simpliest case -- a single remaster operation}

%\subsubsection{Message format (optional)}
% move to implementation part

\subsubsection{BlockLogApp Logic}

\subsubsection{Scheduler Logic}



