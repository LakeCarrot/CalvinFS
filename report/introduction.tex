\section{Introduction}

\note{blue}{copied from proposal}

CalvinFS is a highly scalable distributed filesystem, which stores files metadata in a distributed database rather than in a single server. This design choice dramatically increases the scalability of the filesystem, and as well overcomes single-point failure of metadata server so that provides higher availability. However, such optimization incurs additional overhead on manipulating file metadata. Originally, modifying the metadata of a file only needs a single RPC call from the client to the metadata server, while now it needs a distributed transaction to achieve this goal.

Moreover, in order to fulfill stringent availability requirement, data are often replicated in geographically separated regions, to overcome the entire datacenter outage caused by natural disaster. In such scenario, the cross-region distributed transaction will result in unacceptably high latency. To be specific, in CalvinFS, the latency mainly stems from following two aspects: 1) the global ordering of all distributed transactions to maintain consistency between replicas; 2) Concurrency control when executing transactions, such as Two-phase Locking (2PL).